\documentclass[10pt,conference,compsocconf]{IEEEtran}

%\usepackage{times}
%\usepackage{balance}
\usepackage{url}
\usepackage{graphicx}
\usepackage{natbib}    % For figure environment

\newcommand{\spacing}{\hspace{1cm}}
\begin{document}
    \title{Collaborative Filtering}

    \author{
    Filip Batur\spacing Patrik Okanovic\spacing Rafael Sterzinger\spacing Fatjon Zogaj\\
    ETH Zurich, Switzerland
    }

    \maketitle

    \begin{abstract}

    \end{abstract}


    \section{Introduction}
    %TODO write generally about collaborative filtering/recommender system
    %TODO write about related problem of netflix price and movielense
    %TODO Maybe take a look into "On the difficulta on evaluating Baselines" -> old proposed methods perform surprisingly well, also as information on general methods


    \section{Related Work}
    %TODO might not be necessary only for text filler if needed


    \section{Models and Methods}

    %TODO mention papers with code and the used baselines from there and from the collab notebook
    %TODO Maybe take a look into "On the difficulta on evaluating Baselines"

    %TODO add clipping techniques, initialization of missing values, and if done de-biasing

    \subsection{Singular Value Decomposition}

    \subsection{Non-Negative Matrix Factorization}

    \subsection{Auto Encoder}

    \subsection{Kernel Net}

    \subsection{Bayesian Factorization Machines}
    For our final baseline, we explored Bayesian Factorization Machines \textbf{(BFM)} which are Bayesian variants of the former known Factorization Machines \textbf{(FM)}, introduced by \citeauthor{rendle_factorization_2010}\cite{rendle_factorization_2010}.
    FMs

    \subsubsection{User Features}
    \cite{lee_improving_2017}

    \subsubsection{Movie Features}

    \subsubsection{Iterative SVD}

    \subsubsection{Ensembles}


    \section{Results and Discussion}

    \subsection{Comparison to Baselines}

    You compare your novel algorithm to \emph{at least two baseline
    algorithms}. For the baselines, you can use the implementations you
    developed as part of the programming assignments.


    \section{Conclusion}

    Organize the results section based on the sequence of table and
    figures you include. Prepare the tables and figures as soon as all
    the data are analyzed and arrange them in the sequence that best
    presents your findings in a logical way. A good strategy is to note,
    on a draft of each table or figure, the one or two key results you
    want to address in the text portion of the results.
    The information from the figures is
    summarized in Table.




    \bibliographystyle{IEEEtran}
    \bibliography{bibliography}
\end{document}
