\documentclass[10pt,conference,compsocconf]{IEEEtran}

%\usepackage{times}
%\usepackage{balance}
\usepackage{url}
\usepackage{graphicx}    % For figure environment


\begin{document}
    \title{Collaborative Filtering}

    \author{
    Rafael Sterzinger
    ETH Zurich, Switzerland
    }

    \maketitle

    \begin{abstract}

    \end{abstract}


    \section{Introduction}
    \section{Related Work}


    \section{Models and Methods}

    \subsection{Singular Value Decomposition}

    \subsection{Non-Negative Matrix Factorization}

    \subsection{Auto Encoder}

    \subsection{Kernel Net}

    \subsection{Factorization Machines}

    \subsection{Factorization Machines}
    
    \subsection{Ensembles}


    \section{Results and Discussion}

    \subsection{Comparison to Baselines}

    You compare your novel algorithm to \emph{at least two baseline
    algorithms}. For the baselines, you can use the implementations you
    developed as part of the programming assignments.


    \section{Conclusion}

    Organize the results section based on the sequence of table and
    figures you include. Prepare the tables and figures as soon as all
    the data are analyzed and arrange them in the sequence that best
    presents your findings in a logical way. A good strategy is to note,
    on a draft of each table or figure, the one or two key results you
    want to address in the text portion of the results.
    The information from the figures is
    summarized in Table~\ref{tab:fourier-wavelet}.




    \bibliographystyle{IEEEtran}
    \bibliography{bibliography}
\end{document}
